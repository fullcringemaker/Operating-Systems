\documentclass[a4paper, 14pt]{extarticle}

% Поля
%--------------------------------------
\usepackage{geometry}
\geometry{a4paper,tmargin=2cm,bmargin=2cm,lmargin=3cm,rmargin=1cm}
%--------------------------------------


%Russian-specific packages
%--------------------------------------
\usepackage[T2A]{fontenc}
\usepackage[utf8]{inputenc} 
\usepackage[english, main=russian]{babel}
%--------------------------------------

\usepackage{textcomp}

% Красная строка
%--------------------------------------
\usepackage{indentfirst}               
%--------------------------------------             


%Graphics
%--------------------------------------
\usepackage{graphicx}
\graphicspath{ {./images/} }
\usepackage{wrapfig}
%--------------------------------------

% Полуторный интервал
%--------------------------------------
\linespread{1.3}                    
%--------------------------------------

%Выравнивание и переносы
%--------------------------------------
% Избавляемся от переполнений
\sloppy
% Запрещаем разрыв страницы после первой строки абзаца
\clubpenalty=10000
% Запрещаем разрыв страницы после последней строки абзаца
\widowpenalty=10000
%--------------------------------------

%Списки
\usepackage{enumitem}

%Подписи
\usepackage{caption} 

%Гиперссылки
\usepackage{hyperref}

\hypersetup {
	unicode=true
}

%Рисунки
%--------------------------------------
\DeclareCaptionLabelSeparator*{emdash}{~--- }
\captionsetup[figure]{labelsep=emdash,font=onehalfspacing,position=bottom}
%--------------------------------------

\usepackage{tempora}

%Листинги
%--------------------------------------
\usepackage{listings}
\lstset{
  basicstyle=\ttfamily\footnotesize, 
  %basicstyle=\footnotesize\AnkaCoder,        % the size of the fonts that are used for the code
  breakatwhitespace=false,         % sets if automatic breaks shoulbd only happen at whitespace
  breaklines=true,                 % sets automatic line breaking
  captionpos=t,                    % sets the caption-position to bottom
  inputencoding=utf8,
  frame=single,                    % adds a frame around the code
  keepspaces=true,                 % keeps spaces in text, useful for keeping indentation of code (possibly needs columns=flexible)
  keywordstyle=\bf,       % keyword style
  numbers=left,                    % where to put the line-numbers; possible values are (none, left, right)
  numbersep=5pt,                   % how far the line-numbers are from the code
  xleftmargin=25pt,
  xrightmargin=25pt,
  showspaces=false,                % show spaces everywhere adding particular underscores; it overrides 'showstringspaces'
  showstringspaces=false,          % underline spaces within strings only
  showtabs=false,                  % show tabs within strings adding particular underscores
  stepnumber=1,                    % the step between two line-numbers. If it's 1, each line will be numbered
  tabsize=2,                       % sets default tabsize to 8 spaces
  title=\lstname                   % show the filename of files included with \lstinputlisting; also try caption instead of title
}
%--------------------------------------

%%% Математические пакеты %%%
%--------------------------------------
\usepackage{amsthm,amsfonts,amsmath,amssymb,amscd}  % Математические дополнения от AMS
\usepackage{mathtools}                              % Добавляет окружение multlined
\usepackage[perpage]{footmisc}
%--------------------------------------

%--------------------------------------
%			НАЧАЛО ДОКУМЕНТА
%--------------------------------------

\begin{document}

%--------------------------------------
%			ТИТУЛЬНЫЙ ЛИСТ
%--------------------------------------
\begin{titlepage}
\thispagestyle{empty}
\newpage


%Шапка титульного листа
%--------------------------------------
\vspace*{-60pt}
\hspace{-65pt}
\begin{minipage}{0.3\textwidth}
\hspace*{-20pt}\centering
\includegraphics[width=\textwidth]{emblem}
\end{minipage}
\begin{minipage}{0.67\textwidth}\small \textbf{
\vspace*{-0.7ex}
\hspace*{-6pt}\centerline{Министерство науки и высшего образования Российской Федерации}
\vspace*{-0.7ex}
\centerline{Федеральное государственное бюджетное образовательное учреждение }
\vspace*{-0.7ex}
\centerline{высшего образования}
\vspace*{-0.7ex}
\centerline{<<Московский государственный технический университет}
\vspace*{-0.7ex}
\centerline{имени Н.Э. Баумана}
\vspace*{-0.7ex}
\centerline{(национальный исследовательский университет)>>}
\vspace*{-0.7ex}
\centerline{(МГТУ им. Н.Э. Баумана)}}
\end{minipage}
%--------------------------------------

%Полосы
%--------------------------------------
\vspace{-25pt}
\hspace{-35pt}\rule{\textwidth}{2.3pt}

\vspace*{-20.3pt}
\hspace{-35pt}\rule{\textwidth}{0.4pt}
%--------------------------------------

\vspace{1.5ex}
\hspace{-35pt} \noindent \small ФАКУЛЬТЕТ\hspace{80pt} <<Информатика и системы управления>>

\vspace*{-16pt}
\hspace{47pt}\rule{0.83\textwidth}{0.4pt}

\vspace{0.5ex}
\hspace{-35pt} \noindent \small КАФЕДРА\hspace{50pt} <<Теоретическая информатика и компьютерные технологии>>

\vspace*{-16pt}
\hspace{30pt}\rule{0.866\textwidth}{0.4pt}

\vspace{11em}

\begin{center}
\Large {\bf Отчет о лабораторной работе №4} \\ 
\large {\bf по курсу <<Операционные системы>>} \\
\large «Операции с виртуальной памятью»
\end{center}\normalsize

\vspace{8em}


\begin{flushright}
  {Студент: Трофименко Д. И. \hspace*{15pt}\\ 
  \vspace{2ex}
  Группа: ИУ9 - 42Б \hspace*{15pt}\\
  \vspace{2ex}
  Преподаватель: Брагин А. В.}
\end{flushright}

\bigskip

\vfill
 

\begin{center}
\textsl{Москва, 2025}
\end{center}
\end{titlepage}
%--------------------------------------
%		КОНЕЦ ТИТУЛЬНОГО ЛИСТА
%--------------------------------------

\renewcommand{\ttdefault}{pcr}

\setlength{\tabcolsep}{3pt}
\newpage
\setcounter{page}{2} 
\begin{center}
\section*{Содержание}\label{Sect::task}
\end{center}
\begin{flushleft}
\begin{tabular}{l@{\hspace{4cm}}r}
1) Постановка задачи & \hspace{4cm} \framebox[1cm]{3} \\
2) Практическая реализация & \hspace{4cm} \framebox[1cm]{4} \\
3) Результаты & \hspace{4cm} \framebox[1cm]{7} \\
4) Выводы & \hspace{4cm} \framebox[1cm]{11} \\
5) Список литературы & \hspace{4cm} \framebox[1cm]{12} \\
\end{tabular}
\end{flushleft}

\newpage
\begin{center}
\section{Постановка задачи}
а) ReactOS
\end{center}
Используя созданный в лабораторной работе № 2 минимальный драйвер, совместимый с ОС Windows NT/ReactOS и NetBSD, реализовать следующие действия с виртуальной памятью.\\
1. Зарезервировать  10  страниц виртуальной памяти,  используя  функцию ZwAllocateVirtualMemory(NtCurrentProcess(), MEM\_RESERVE, …)..\\
2. Обеспечить пять первых страниц из выделенных 10 физическими  страницами  памяти,  используя  ZwAllocateVirtualMemory (NtCurrentProcess(), MEM\_COMMIT, …).\\
3. Вывести физические адреса и значения PTE для всех 10 страниц в шестнадцатеричном формате.\\
4. Освободить выделенную память.

\begin{center}
б) NetBSD
\end{center}
Используя созданный в лабораторной работе № 2 драйвер, реализовать следующие действия с виртуальной памятью.\\
1. Зарезервировать 10 страниц виртуальной памяти, используя функцию ядра uvm\_km\_alloc.\\
2. Обеспечить пять первых страниц из выделенных 10 физическими страницами памяти, используя функции ядра uvm\_pglistalloc и pmap\_kenter\_pa.\\
3. Вывести физические адреса и значения PTE для всех 10 страниц в шестнадцатеричном формате.\\
4.  Освободить  выделенную  память, используя функции  ядра pmap\_kremove, pmap\_update, uvm\_pglistfree, uvm\_km\_free.\\
Драйвер должен выгружаться динамически с помощью команды modunload


\vspace{0.1em}


\newpage
\vspace{2em}
\begin{center}
\section{Практическая реализация}\label{Sect::res}
а) ReactOS
\end{center}
1) Изучение общей структуры загружаемого модуля ядра в системе
ReactOS в официальной документации. Изучение библиотек,
позволяющим получить доступ к виртуальной памяти\\
2) Модификация исходных файлов операционной системы, путём
добавления в папку drivers файла с исходным кодом драйвера, а
также служебных файлов, необходимых для компиляции\\
3) Пересборка iso-образа операционной системы и его установка на
виртуальную машину Oracle VirtualBox.\\
4) Запуск полученного драйвера с помощью встроенной утилиты sc start.\\

\begin{center}
б) NetBSD
\end{center}
1) Изучение общей структуры загружаемого модуля ядра в системе
NetBSD в официальной документации. Изучение библиотек,
позволяющим получить доступ к списку процессов в системе\\
2) Добавление исходного кода драйвера в папку /usr/src/sys/dev и
makefile в папку /usr/src/sys/modules/lab3.\\
3) Сборка драйвера с помощью утилиты make и его установка с
помощью встроенной утилиты modload.\\

\newpage

Исходный код программы представлен в листингах~\ref{lst:code1}--~\ref{lst:code2}.
\begin{figure}[!htb]
\begin{lstlisting}[language={},caption={
lab4driver.c (ReactOS).},label={lst:code1}]
...
DRIVER_UNLOAD DriverUnload;
VOID NTAPI DriverUnload(IN PDRIVER_OBJECT DriverObject)
{
    DPRINT1("------------------DRIVER-UNLOADED--------------------\n");
    IoDeleteDevice(DriverObject->DeviceObject);
}
NTSTATUS NTAPI DriverEntry(IN PDRIVER_OBJECT DriverObject,
            IN PUNICODE_STRING RegistryPath)
{
    DPRINT1("---------------Trofimenko Dmitriy---------------\n");
    DriverObject->DriverUnload = DriverUnload;
    MmPageEntireDriver(DriverEntry);
    ...
    ZwAllocateVirtualMemory(NtCurrentProcess(), &Pages, 0, &sizeReserve, MEM_RESERVE, PAGE_READWRITE);
    DPRINT1("10 PAGES RESERVED\n");
    ZwAllocateVirtualMemory(NtCurrentProcess(), &Pages, 0 , &sizeCommit, MEM_COMMIT, PAGE_READWRITE);
    DPRINT1("5 PAGES COMMITED\n");
    for(i = 0; i < 5; i++) {
        *((PCHAR)Pages + 0x1000 * i) = i + 1;
    }
    for(i = 0; i < 10; i++) {
        pte = ((ULONG)Pages >> 12) + PTE_BASE + i;
        DPRINT1("Page: %d\n\
            Physical address: %X\n\
            Valid:            %d\n\
            Accessed:         %d\n\
            Dirty:            %d\n\
            \n", i + 1, 
            pte->Valid ? (pte->PageFrameNumber << 12) : 0,
            pte->Valid,
            pte->Accessed,
            pte->Dirty);
    }
    ZwFreeVirtualMemory(NtCurrentProcess(), &Pages, &sizeCommit, MEM_DECOMMIT);
    DPRINT1("MEMORY IS DECOMMITED\n");
    ZwFreeVirtualMemory(NtCurrentProcess(), &Pages, 0, MEM_RELEASE);
    DPRINT1("MEMORY IS RELEASED\n");
    return STATUS_SUCCESS;
}

\end{lstlisting}
\end{figure}

\begin{figure}[!htb]
\begin{lstlisting}[language={},caption={
lab4.c (NetBSD).},label={lst:code2}]
...
MODULE(MODULE_CLASS_MISC, lab4, NULL);
#define NUM_PAGES 10
static struct vm_page *pglist = NULL;
static void print_page_info(vaddr_t va, int page_num) {
    ...
    printf("Page - %d\n", page_num);
    printf("Valid - %s\n", valid ? "true" : "false");
    printf("Used - %s\n", used ? "true" : "false");
    printf("Modified - %s\n", modified ? "true" : "false");
    printf("Physical address - 0x%08lx\n\n", (unsigned long)pa);
}
static int lab4_modcmd(modcmd_t cmd, void *arg) {
    ...
    if (cmd == MODULE_CMD_INIT) {
        addr = uvm_km_alloc(kernel_map, size, PAGE_SIZE, UVM_KMF_VAONLY);
        ...
        int r = uvm_pglistalloc(5 * PAGE_SIZE, 0, 0xffffffff, 0, 0, &mlist, 5, 0);
        if (r != 0) {
            uvm_km_free(kernel_map, addr, size, 0);
            printf("lab4: physical memory allocation failed\n");
            return ENOMEM;
        }
        pglist = TAILQ_FIRST(&mlist); 
        struct vm_page *pg = TAILQ_FIRST(&mlist);
        for (int i = 0; i < 5 && pg != NULL; i++) {
            paddr_t pa = VM_PAGE_TO_PHYS(pg);
            pmap_kenter_pa(addr + i * PAGE_SIZE, pa, VM_PROT_READ | VM_PROT_WRITE, 0);
            pg = TAILQ_NEXT(pg, pageq.queue);
        }
        pmap_update(pmap_kernel());
        printf("Before free:\n");
        ...
        pmap_kremove(addr, size);
        pmap_update(pmap_kernel());
        if (pglist)
            uvm_pglistfree(&mlist);
        uvm_km_free(kernel_map, addr, size, 0);
        return 0;
    }
    if (cmd == MODULE_CMD_FINI) {
        printf("lab4: module unloaded\n");
        return 0;
    }
    return ENOTTY;
}


\end{lstlisting}
\end{figure}


\newpage

\vspace{1em}
\begin{center}
\section{Результаты}

\begin{minipage}{\linewidth}
\vspace{50pt}
\centering
\includegraphics[width=1.0\linewidth]{result_reactos (1).jpg}
\captionof{figure}{result\_reactos (1)}
\label{fig:result_reactos (1).jpg}
\end{minipage}

\begin{minipage}{\linewidth}
\vspace{20pt}
\centering
\includegraphics[width=1.0\linewidth]{result_reactos (2).jpg}
\captionof{figure}{result\_reactos (2)}
\label{fig:result_reactos (2).jpg}
\end{minipage}

\begin{minipage}{\linewidth}
\vspace{50pt}
\centering
\includegraphics[width=1.0\linewidth]{result_netbsd (1).jpg}
\captionof{figure}{result\_netbsd (1)}
\label{fig:result_netbsd (1).jpg}
\end{minipage}

\begin{minipage}{\linewidth}
\vspace{50pt}
\centering
\includegraphics[width=1.0\linewidth]{result_netbsd (2).jpg}
\captionof{figure}{result\_netbsd (2)}
\label{fig:result_netbsd (2).jpg}
\end{minipage}

\end{center}

\newpage

\begin{center}
\section{Выводы}
а) ReactOS
\end{center} 
В ходе выполнения работы были изучены механизмы управления виртуальной памятью в режиме ядра Windows NT/ReactOS, включая резервирование и выделение физических страниц с помощью функций ZwAllocateVirtualMemory. Основной сложностью стало корректное определение и вывод физических адресов и значений PTE (Page Table Entry), так как это требует глубокого понимания архитектуры виртуальной памяти и работы MMU (Memory Management Unit). Однако сам процесс резервирования и коммита памяти оказался достаточно простым благодаря четкой документации API ядра. Также было замечено, что ReactOS, несмотря на свою неполную совместимость с Windows NT, корректно обрабатывает базовые операции с памятью, что подтверждает работоспособность её подсистемы управления памятью. Освобождение ресурсов прошло без осложнений, что подтвердило важность правильного завершения работы драйвера во избежание утечек памяти.

\begin{center}
б) NetBSD
\end{center}
В ходе работы были изучены механизмы управления виртуальной памятью в режиме ядра NetBSD, включая резервирование памяти через uvm\_km\_alloc и выделение физических страниц с помощью uvm\_pglistalloc и pmap\_kenter\_pa. Основной сложностью стало корректное взаимодействие с подсистемой UVM (Virtual Memory Manager) и pmap (physical mapping), так как NetBSD использует иной подход к управлению памятью по сравнению с Windows NT. Однако процесс освобождения памяти через pmap\_kremove, pmap\_update, uvm\_pglistfree и uvm\_km\_free оказался логичным и предсказуемым. Динамическая загрузка и выгрузка модуля с помощью modload/modunload прошла успешно, что подтвердило стабильность работы драйвера. В целом, работа с памятью в NetBSD потребовала более глубокого понимания архитектуры ядра, но предоставила больший контроль над управлением страницами.



\begin{center}
\section{Список литературы}
\end{center}
1) Учебно-методическое пособие "Операционные системы" [Электронный ресурс]. URL:\\
https://press.bmstu.ru/catalog/item/8226/reader/\\


\end{document}
