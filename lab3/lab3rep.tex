\documentclass[a4paper, 14pt]{extarticle}

% Поля
%--------------------------------------
\usepackage{geometry}
\geometry{a4paper,tmargin=2cm,bmargin=2cm,lmargin=3cm,rmargin=1cm}
%--------------------------------------


%Russian-specific packages
%--------------------------------------
\usepackage[T2A]{fontenc}
\usepackage[utf8]{inputenc} 
\usepackage[english, main=russian]{babel}
%--------------------------------------

\usepackage{textcomp}

% Красная строка
%--------------------------------------
\usepackage{indentfirst}               
%--------------------------------------             


%Graphics
%--------------------------------------
\usepackage{graphicx}
\graphicspath{ {./images/} }
\usepackage{wrapfig}
%--------------------------------------

% Полуторный интервал
%--------------------------------------
\linespread{1.3}                    
%--------------------------------------

%Выравнивание и переносы
%--------------------------------------
% Избавляемся от переполнений
\sloppy
% Запрещаем разрыв страницы после первой строки абзаца
\clubpenalty=10000
% Запрещаем разрыв страницы после последней строки абзаца
\widowpenalty=10000
%--------------------------------------

%Списки
\usepackage{enumitem}

%Подписи
\usepackage{caption} 

%Гиперссылки
\usepackage{hyperref}

\hypersetup {
	unicode=true
}

%Рисунки
%--------------------------------------
\DeclareCaptionLabelSeparator*{emdash}{~--- }
\captionsetup[figure]{labelsep=emdash,font=onehalfspacing,position=bottom}
%--------------------------------------

\usepackage{tempora}

%Листинги
%--------------------------------------
\usepackage{listings}
\lstset{
  basicstyle=\ttfamily\footnotesize, 
  %basicstyle=\footnotesize\AnkaCoder,        % the size of the fonts that are used for the code
  breakatwhitespace=false,         % sets if automatic breaks shoulbd only happen at whitespace
  breaklines=true,                 % sets automatic line breaking
  captionpos=t,                    % sets the caption-position to bottom
  inputencoding=utf8,
  frame=single,                    % adds a frame around the code
  keepspaces=true,                 % keeps spaces in text, useful for keeping indentation of code (possibly needs columns=flexible)
  keywordstyle=\bf,       % keyword style
  numbers=left,                    % where to put the line-numbers; possible values are (none, left, right)
  numbersep=5pt,                   % how far the line-numbers are from the code
  xleftmargin=25pt,
  xrightmargin=25pt,
  showspaces=false,                % show spaces everywhere adding particular underscores; it overrides 'showstringspaces'
  showstringspaces=false,          % underline spaces within strings only
  showtabs=false,                  % show tabs within strings adding particular underscores
  stepnumber=1,                    % the step between two line-numbers. If it's 1, each line will be numbered
  tabsize=2,                       % sets default tabsize to 8 spaces
  title=\lstname                   % show the filename of files included with \lstinputlisting; also try caption instead of title
}
%--------------------------------------

%%% Математические пакеты %%%
%--------------------------------------
\usepackage{amsthm,amsfonts,amsmath,amssymb,amscd}  % Математические дополнения от AMS
\usepackage{mathtools}                              % Добавляет окружение multlined
\usepackage[perpage]{footmisc}
%--------------------------------------

%--------------------------------------
%			НАЧАЛО ДОКУМЕНТА
%--------------------------------------

\begin{document}

%--------------------------------------
%			ТИТУЛЬНЫЙ ЛИСТ
%--------------------------------------
\begin{titlepage}
\thispagestyle{empty}
\newpage


%Шапка титульного листа
%--------------------------------------
\vspace*{-60pt}
\hspace{-65pt}
\begin{minipage}{0.3\textwidth}
\hspace*{-20pt}\centering
\includegraphics[width=\textwidth]{emblem}
\end{minipage}
\begin{minipage}{0.67\textwidth}\small \textbf{
\vspace*{-0.7ex}
\hspace*{-6pt}\centerline{Министерство науки и высшего образования Российской Федерации}
\vspace*{-0.7ex}
\centerline{Федеральное государственное бюджетное образовательное учреждение }
\vspace*{-0.7ex}
\centerline{высшего образования}
\vspace*{-0.7ex}
\centerline{<<Московский государственный технический университет}
\vspace*{-0.7ex}
\centerline{имени Н.Э. Баумана}
\vspace*{-0.7ex}
\centerline{(национальный исследовательский университет)>>}
\vspace*{-0.7ex}
\centerline{(МГТУ им. Н.Э. Баумана)}}
\end{minipage}
%--------------------------------------

%Полосы
%--------------------------------------
\vspace{-25pt}
\hspace{-35pt}\rule{\textwidth}{2.3pt}

\vspace*{-20.3pt}
\hspace{-35pt}\rule{\textwidth}{0.4pt}
%--------------------------------------

\vspace{1.5ex}
\hspace{-35pt} \noindent \small ФАКУЛЬТЕТ\hspace{80pt} <<Информатика и системы управления>>

\vspace*{-16pt}
\hspace{47pt}\rule{0.83\textwidth}{0.4pt}

\vspace{0.5ex}
\hspace{-35pt} \noindent \small КАФЕДРА\hspace{50pt} <<Теоретическая информатика и компьютерные технологии>>

\vspace*{-16pt}
\hspace{30pt}\rule{0.866\textwidth}{0.4pt}

\vspace{11em}

\begin{center}
\Large {\bf Отчет о лабораторной работе №3} \\ 
\large {\bf по курсу <<Операционные системы>>} \\
\large «Список процессов в системе»
\end{center}\normalsize

\vspace{8em}


\begin{flushright}
  {Студент: Трофименко Д. И. \hspace*{15pt}\\ 
  \vspace{2ex}
  Группа: ИУ9 - 42Б \hspace*{15pt}\\
  \vspace{2ex}
  Преподаватель: Брагин А. В.}
\end{flushright}

\bigskip

\vfill
 

\begin{center}
\textsl{Москва, 2025}
\end{center}
\end{titlepage}
%--------------------------------------
%		КОНЕЦ ТИТУЛЬНОГО ЛИСТА
%--------------------------------------

\renewcommand{\ttdefault}{pcr}

\setlength{\tabcolsep}{3pt}
\newpage
\setcounter{page}{2} 
\begin{center}
\section*{Содержание}\label{Sect::task}
\end{center}
\begin{flushleft}
\begin{tabular}{l@{\hspace{4cm}}r}
1) Постановка задачи & \hspace{4cm} \framebox[1cm]{3} \\
2) Практическая реализация & \hspace{4cm} \framebox[1cm]{4} \\
3) Результаты & \hspace{4cm} \framebox[1cm]{6} \\
4) Выводы & \hspace{4cm} \framebox[1cm]{8} \\
5) Список литературы & \hspace{4cm} \framebox[1cm]{8} \\
\end{tabular}
\end{flushleft}
\newpage

\begin{center}
\section{Постановка задачи}
а) ReactOS
\end{center}
На основе драйвера из лабораторной работы № 2 создать новый драйвер, который выводит в отладочный лог список процессов  в  системе, используя API  ядра.  Список  должен  состоять  из имени процесса (имя образа — Image Name), идентификатора процесса (PID) и идентификатора родительского процесса.\\  Драйвер  должен загружаться  и  выгружаться по требованию так, чтобы можно было проверить  соответствие списков процессов, выведенных драйвером, и с помощью утилиты Task Manager (taskmgr.exe).\\
Информацию о процессах получить из ядра с использованием функции  ZwQuerySystemInformation(SystemProcessInformation, …). Выделение и освобождение памяти для информации о процессах реализовать,  используя  функции  ExAllocatePool(PagedPool, …) и ExFreePool(). Предусмотреть ситуацию, в  которой запрошенный размер буфера, необходимый для размещения информации о про-цессах,  может  измениться  в большую  сторону  между  вызовами  функции ZwQuerySystemInformation, и в этом случае произвести новое выделение памяти с освобождением старого буфера.

\begin{center}
б) NetBSD
\end{center}
На основе драйвера из лабораторной работы № 2 создать новый драйвер, который выводит в отладочный лог список процессов  в  системе, используя API  ядра.  Список должен  состоять  из имени процесса (имя образа — Image Name), идентификатора процесса (PID) и идентификатора родительского процесса.\\
Драйвер должен загружаться и выгружаться по требованию так, чтобы  можно  было  проверить соответствие  списков  процессов, выведенных драйвером, и с помощью утилиты ps.\\
Информацию  о  процессах получить  из  ядра,  используя  глобальную переменную allproc. Для этого необходимо включить за-головочные файлы:\\
\#include <sys/module.h>\\
\#include <sys/kernel.h>\\
\#include <sys/proc.h>\\
Самостоятельно следует изучить проблему защиты от «грязно-го чтения» списка процессов в момент его изменения операционной системой (например, при создании нового процесса или удалении завершенного).


\vspace{0.1em}


\vspace{2em}
\begin{center}
\section{Практическая реализация}\label{Sect::res}
а) ReactOS
\end{center}
1) Изучение общей структуры загружаемого модуля ядра в системе
ReactOS в официальной документации. Изучение библиотек,
позволяющим получить доступ к виртуальной памяти\\
2) Модификация исходных файлов операционной системы, путём
добавления в папку drivers файла с исходным кодом драйвера, а
также служебных файлов, необходимых для компиляции\\
3) Пересборка iso-образа операционной системы и его установка на
виртуальную машину Oracle VirtualBox.\\
4) Запуск полученного драйвера с помощью встроенной утилиты sc start.\\

\begin{center}
б) NetBSD
\end{center}
1) Изучение общей структуры загружаемого модуля ядра в системе
NetBSD в официальной документации. Изучение библиотек,
позволяющим получить доступ к списку процессов в системе\\
2) Добавление исходного кода драйвера в папку /usr/src/sys/dev и
makefile в папку /usr/src/sys/modules/lab3.\\
3) Сборка драйвера с помощью утилиты make и его установка с
помощью встроенной утилиты modload.\\

\newpage

Исходный код программы представлен в листингах~\ref{lst:code1}--~\ref{lst:code2}.
\begin{figure}[!htb]
\begin{lstlisting}[language={},caption={
lab3driver.c (ReactOS).},label={lst:code1}]
...
DriverEntry(IN PDRIVER_OBJECT DriverObject, IN PUNICODE_STRING RegistryPath) {
    ...
    DPRINT1("---------------Trofimenko Dmitriy---------------\n");
    ...
    PVOID process;
    while (0 == 0) {
        process = ExAllocatePoolWithTag(PagedPool, buffsz, 'Proc');
        if (!process) {
            return STATUS_MEMORY_NOT_ALLOCATED;
        }
        ULONG buffSize = 0;
        status = ZwQuerySystemInformation(SystemProcessInformation, process, buffsz, &buffSize);
        if (buffSize != buffsz) {
            ExFreePool(process);
            buffsz = buffSize;
            continue;
        }
        if (NT_ERROR(status)) {
            ExFreePool(process);
            return status;
        }
        break;
    }
    PSYSTEM_PROCESS_INFORMATION curtpr = (PSYSTEM_PROCESS_INFORMATION)process;
    while (curtpr) {
        RtlInitUnicodeString(&imagename, curtpr->ImageName.Buffer);
        DPRINT1("Name: %wZ, PID: %d, Parent PID: %d\n", 
                    &imagename, 
                    (ULONG)curtpr->UniqueProcessId,
                    (ULONG)curtpr->InheritedFromUniqueProcessId);

        if (!curtpr->NextEntryOffset) {
            break;
        }
        curtpr = (PSYSTEM_PROCESS_INFORMATION)((PUCHAR)curtpr + curtpr->NextEntryOffset);
    }
    ... 
}


\end{lstlisting}
\end{figure}

\begin{figure}[!htb]
\begin{lstlisting}[language={},caption={
lab3.c (NetBSD).},label={lst:code2}]
...
MODULE(MODULE_CLASS_MISC, lab3, NULL);
static int lab3_modcmd(modcmd_t cmd, void *arg) {
    printf("---------------Trofimenko Dmitriy---------------\n");
    struct proc *p;
    switch (cmd) {
        case MODULE_CMD_INIT:
            PROCLIST_FOREACH(p, &allproc) {
                printf("%d %s ", p->p_pid, p->p_comm);
                if (p->p_pptr != NULL) {
                    printf("%d\n", p->p_pptr->p_pid);
                } else {
                    printf("\n");
                }
            }
            break;
        case MODULE_CMD_FINI:
            break;
        default:
            return ENOTTY;
        }
    return 0;
}


\end{lstlisting}
\end{figure}




\newpage
\vspace{1em}
\begin{center}
\section{Результаты}

\begin{minipage}{\linewidth}
\vspace{50pt}
\centering
\includegraphics[width=1.0\linewidth]{result_reactos.jpg}
\captionof{figure}{result\_reactos}
\label{fig:result_reactos.jpg}
\end{minipage}

\begin{minipage}{\linewidth}
\vspace{50pt}
\centering
\includegraphics[width=1.0\linewidth]{result_netbsd.jpg}
\captionof{figure}{result\_netbsd}
\label{fig:result_netbsd.jpg}
\end{minipage}

\end{center}

\newpage

\begin{center}
\section{Выводы}
а) ReactOS
\end{center}
В ходе выполнения лабораторной работы были изучены методы взаимодействия с API ядра ReactOS для получения информации о процессах, включая использование функции ZwQuerySystemInformation с параметром SystemProcessInformation, а также управление памятью через ExAllocatePool и ExFreePool. Основной сложностью стало корректное определение размера буфера для данных о процессах, так как их количество может динамически изменяться, что потребовало повторного выделения памяти. Однако работа с отладочным выводом в ReactOS оказалась удобной, что позволило быстро проверить соответствие списка процессов данным из Task Manager. Также было подтверждено, что загрузка и выгрузка драйвера через sc работают стабильно, что упрощает тестирование функциональности.

\begin{center}
б) NetBSD
\end{center}
В ходе выполнения лабораторной работы в NetBSD были изучены методы работы с глобальным списком процессов allproc и особенности безопасного доступа к нему с учетом возможных изменений в многозадачной среде. Основной сложностью стала реализация защиты от «грязного чтения» списка процессов, так как он может изменяться во время итерации, что потребовало использования механизмов синхронизации, таких как блокировка proc\_lock. Однако работа с макросом PROCLIST\_FOREACH значительно упростила обход списка процессов, а загрузка модуля через modload оказалась удобной и быстрой. Сравнение вывода драйвера с утилитой ps подтвердило корректность полученных данных, что позволило убедиться в правильности реализации.

\begin{center}
\section{Список литературы}
\end{center}
1) Учебно-методическое пособие "Операционные системы" [Электронный ресурс]. URL:\\
https://press.bmstu.ru/catalog/item/8226/reader/\\


\end{document}
