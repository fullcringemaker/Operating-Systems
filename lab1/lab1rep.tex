\documentclass[a4paper, 14pt]{extarticle}

% Поля
%--------------------------------------
\usepackage{geometry}
\geometry{a4paper,tmargin=2cm,bmargin=2cm,lmargin=3cm,rmargin=1cm}
%--------------------------------------


%Russian-specific packages
%--------------------------------------
\usepackage[T2A]{fontenc}
\usepackage[utf8]{inputenc} 
\usepackage[english, main=russian]{babel}
%--------------------------------------

\usepackage{textcomp}

% Красная строка
%--------------------------------------
\usepackage{indentfirst}               
%--------------------------------------             


%Graphics
%--------------------------------------
\usepackage{graphicx}
\graphicspath{ {./images/} }
\usepackage{wrapfig}
%--------------------------------------

% Полуторный интервал
%--------------------------------------
\linespread{1.3}                    
%--------------------------------------

%Выравнивание и переносы
%--------------------------------------
% Избавляемся от переполнений
\sloppy
% Запрещаем разрыв страницы после первой строки абзаца
\clubpenalty=10000
% Запрещаем разрыв страницы после последней строки абзаца
\widowpenalty=10000
%--------------------------------------

%Списки
\usepackage{enumitem}

%Подписи
\usepackage{caption} 

%Гиперссылки
\usepackage{hyperref}

\hypersetup {
	unicode=true
}

%Рисунки
%--------------------------------------
\DeclareCaptionLabelSeparator*{emdash}{~--- }
\captionsetup[figure]{labelsep=emdash,font=onehalfspacing,position=bottom}
%--------------------------------------

\usepackage{tempora}

%Листинги
%--------------------------------------
\usepackage{listings}
\lstset{
  basicstyle=\ttfamily\footnotesize, 
  %basicstyle=\footnotesize\AnkaCoder,        % the size of the fonts that are used for the code
  breakatwhitespace=false,         % sets if automatic breaks shoulbd only happen at whitespace
  breaklines=true,                 % sets automatic line breaking
  captionpos=t,                    % sets the caption-position to bottom
  inputencoding=utf8,
  frame=single,                    % adds a frame around the code
  keepspaces=true,                 % keeps spaces in text, useful for keeping indentation of code (possibly needs columns=flexible)
  keywordstyle=\bf,       % keyword style
  numbers=left,                    % where to put the line-numbers; possible values are (none, left, right)
  numbersep=5pt,                   % how far the line-numbers are from the code
  xleftmargin=25pt,
  xrightmargin=25pt,
  showspaces=false,                % show spaces everywhere adding particular underscores; it overrides 'showstringspaces'
  showstringspaces=false,          % underline spaces within strings only
  showtabs=false,                  % show tabs within strings adding particular underscores
  stepnumber=1,                    % the step between two line-numbers. If it's 1, each line will be numbered
  tabsize=2,                       % sets default tabsize to 8 spaces
  title=\lstname                   % show the filename of files included with \lstinputlisting; also try caption instead of title
}
%--------------------------------------

%%% Математические пакеты %%%
%--------------------------------------
\usepackage{amsthm,amsfonts,amsmath,amssymb,amscd}  % Математические дополнения от AMS
\usepackage{mathtools}                              % Добавляет окружение multlined
\usepackage[perpage]{footmisc}
%--------------------------------------

%--------------------------------------
%			НАЧАЛО ДОКУМЕНТА
%--------------------------------------

\begin{document}

%--------------------------------------
%			ТИТУЛЬНЫЙ ЛИСТ
%--------------------------------------
\begin{titlepage}
\thispagestyle{empty}
\newpage


%Шапка титульного листа
%--------------------------------------
\vspace*{-60pt}
\hspace{-65pt}
\begin{minipage}{0.3\textwidth}
\hspace*{-20pt}\centering
\includegraphics[width=\textwidth]{emblem}
\end{minipage}
\begin{minipage}{0.67\textwidth}\small \textbf{
\vspace*{-0.7ex}
\hspace*{-6pt}\centerline{Министерство науки и высшего образования Российской Федерации}
\vspace*{-0.7ex}
\centerline{Федеральное государственное бюджетное образовательное учреждение }
\vspace*{-0.7ex}
\centerline{высшего образования}
\vspace*{-0.7ex}
\centerline{<<Московский государственный технический университет}
\vspace*{-0.7ex}
\centerline{имени Н.Э. Баумана}
\vspace*{-0.7ex}
\centerline{(национальный исследовательский университет)>>}
\vspace*{-0.7ex}
\centerline{(МГТУ им. Н.Э. Баумана)}}
\end{minipage}
%--------------------------------------

%Полосы
%--------------------------------------
\vspace{-25pt}
\hspace{-35pt}\rule{\textwidth}{2.3pt}

\vspace*{-20.3pt}
\hspace{-35pt}\rule{\textwidth}{0.4pt}
%--------------------------------------

\vspace{1.5ex}
\hspace{-35pt} \noindent \small ФАКУЛЬТЕТ\hspace{80pt} <<Информатика и системы управления>>

\vspace*{-16pt}
\hspace{47pt}\rule{0.83\textwidth}{0.4pt}

\vspace{0.5ex}
\hspace{-35pt} \noindent \small КАФЕДРА\hspace{50pt} <<Теоретическая информатика и компьютерные технологии>>

\vspace*{-16pt}
\hspace{30pt}\rule{0.866\textwidth}{0.4pt}

\vspace{11em}

\begin{center}
\Large {\bf Отчет о лабораторной работе №1} \\ 
\large {\bf по курсу <<Операционные системы>>} \\
\large «ReactOS и NetBSD. Среда сборки, установка и тестирование в виртуальной машине»
\end{center}\normalsize

\vspace{8em}


\begin{flushright}
  {Студент: Трофименко Д. И. \hspace*{15pt}\\ 
  \vspace{2ex}
  Группа: ИУ9 - 42Б \hspace*{15pt}\\
  \vspace{2ex}
  Преподаватель: Брагин А. В.}
\end{flushright}

\bigskip

\vfill
 

\begin{center}
\textsl{Москва, 2025}
\end{center}
\end{titlepage}
%--------------------------------------
%		КОНЕЦ ТИТУЛЬНОГО ЛИСТА
%--------------------------------------

\renewcommand{\ttdefault}{pcr}

\setlength{\tabcolsep}{3pt}
\newpage
\setcounter{page}{2} 
\begin{center}
\section*{Содержание}\label{Sect::task}
\end{center}

\begin{flushleft}
\begin{tabular}{l@{\hspace{4cm}}r}
1) Постановка задачи & \hspace{4cm} \framebox[1cm]{3} \\
2) Практическая реализация & \hspace{4cm} \framebox[1cm]{4} \\
3) Результаты & \hspace{4cm} \framebox[1cm]{6} \\
4) Выводы & \hspace{4cm} \framebox[1cm]{8} \\
5) Список литературы & \hspace{4cm} \framebox[1cm]{8} \\
\end{tabular}
\end{flushleft}

\newpage
\begin{center}
\section{Постановка задачи}
а) ReactOS
\end{center}
Настроить среду сборки и тестирования ReactOS на своем компьютере (или на компьютере в учебной аудитории Университета), собрать установочный образ и произвести его установку в виртуальную машину с помощью программных средств виртуализации

\begin{center}
б) NetBSD
\end{center}
Установить новый выпуск NetBSD из дистрибутива с официального сайта  проекта  NetBSD  https://www.netbsd.org/ в  виртуальную машину. Настроить среду сборки и тестирования NetBSD в вирту-альной машине на своем компьютере (или на компьютере в учебной аудитории Университета). Пересобрать ядро в этой виртуальной машине, добавив вывод фамилии обучающегося в debug log.\\
Не следует предпринимат попытки кросс-компилировать ядро NetBSD в Windows, Linux или OS X. Собирать его нужно внутри самой NetBSD, которая установлена в виртуальной машине.

\vspace{0.1em}


\newpage
\vspace{2em}
\begin{center}
\section{Практическая реализация}\label{Sect::res}
а) ReactOS
\end{center}
1) Установил RosBE. С её помощью произвел сборку установочного
образа из исходного кода, который клонировал с официального гит
репозитория reactos (система контроля версий git уже предварительно
у меня на компьютере стояла), используя инструкции из [5].\\
2) Скачала VirtualBox с официального сайта. Проблем с установкой не
возникло.\\
3) Следуя инструкциям из [2], cоздал виртуальную машину для reactos,
настроил вывод com-порта в файл reactos\_debug.log на основной системе.\\
4) Произвел установку собранного в пункте 2 образа на виртуальную
машину. Загрузил её в режиме вывода bootlog в com-порт.\\
5) Проанализировал bootlog. В исходниках reactos в файле,
расположенном по адресу /ntoskrln/kd64/kdinit.c, добавил строчку,
выводящую мою фамилию при загрузке системы с помощью команды
DPRINT().\\
6) Заменил ядро, перезагрузил машину, убедился, что в отладочном логе
появляется фамилия.\\

\begin{center}
б) NetBSD
\end{center}
1) С официального сайта NetBSD скачал установочный образ NetBSD-10.1
для архитектуры amd64.\\
2) В VirtualBox создал машину для NetBSD, настроил сетевое соединение
и вывод com-порта в файл netbsd\_debug.log.\\
3) Установил операционную систему из скачанного образа. Во время
установки произвел автоматическую настройку сети.\\
4) После установки с помощью ftp, пользуясь [4], скачал исходные файлы
ядра NetBSD-10.1.\\
5) Внёс изменения в файл /usr/src/sys/kern/init\_main.c, добавив вывод
своего ФИО с помощью команды printf\\
6) Следуя инструкциям [5], cоздал свою конфигурацию системы,
выполнив команду cp GENERIC izmenenieIshodnikov. Произвел
компиляцию ядра, заменил старое ядро на новое.\\
7) Перезагрузил систему, с помощью команды dmesg | grep Trofimenko
убедился, что мое ФИО выводится в логе.\\
8) Настроил вывод com-порта в файл netbsd\_debug.log на основной системе.\\
9) Перезагрузил машину, убедился, что в отладочном логе
появляется фамилия.\\


Исходный код программ представлен в листингах~\ref{lst:code1}--~\ref{lst:code2}.
\begin{figure}[!htb]
\begin{lstlisting}[language={},caption={
kdinit.c (ReactOS).},label={lst:code1}]
...
KdpPrintBanner(VOID)
{
    SIZE_T MemSizeMBs = KdpGetMemorySizeInMBs(KeLoaderBlock);
    DPRINT1("-----------------------------------------------------\n");
    DPRINT1("INit ReactOS.\n");
    DPRINT1("---------------Trofimenko Dmitriy---------------\n");
    DPRINT1("ReactOS " KERNEL_VERSION_STR " (Build " KERNEL_VERSION_BUILD_STR ") (Commit " KERNEL_VERSION_COMMIT_HASH ")\n");
    DPRINT1("%u System Processor [%u MB Memory]\n", KeNumberProcessors, MemSizeMBs);
    ...
}
...

\end{lstlisting}
\end{figure}

\begin{figure}[!htb]
\begin{lstlisting}[language={},caption={
init\_main.c (NetBSD).},label={lst:code2}]
...
void 
main(void)
{
printf("Trofimenko Dmitriy\n");
...
}
...

\end{lstlisting}
\end{figure}


\newpage


\vspace{1em}
\begin{center}
\section{Результаты}
\begin{minipage}{\linewidth}
\vspace{50pt}
\centering
\includegraphics[width=1.0\linewidth]{reactos.jpg}
\captionof{figure}{reactos}
\label{fig:reactos.jpg}


\end{minipage}
\begin{minipage}{\linewidth}
\vspace{50pt}
\centering
\includegraphics[width=1.0\linewidth]{result_reactos.jpg}
\captionof{figure}{result\_reactos}
\label{fig:result_reactos.jpg}
\end{minipage}

\begin{minipage}{\linewidth}
\vspace{50pt}
\centering
\includegraphics[width=1.0\linewidth]{netbsd.jpg}
\captionof{figure}{netbsd}
\label{fig:netbsd.jpg}
\end{minipage}


\begin{minipage}{\linewidth}
\vspace{50pt}
\centering
\includegraphics[width=1.0\linewidth]{result_netbsd.jpg}
\captionof{figure}{result\_netbsd}
\label{fig:result_netbsd.jpg}
\end{minipage}

\end{center}

\newpage

\begin{center}
\section{Выводы}
а) ReactOS
\end{center}
В ходе работы были приобретены навыки настройки среды сборки ReactOS (RosBE), работы с системой контроля версий Git и процесса компиляции исходного кода в установочный ISO-образ. Основной сложностью стало корректное конфигурирование виртуальной машины для вывода отладочной информации, а также поиск подходящего места в коде ядра (ntoskrnl) для внедрения пользовательской строки с фамилией. Однако сам процесс модификации ядра оказался достаточно простым благодаря подробной документации ReactOS и логичной структуре исходного кода. Важным итогом стало понимание этапов сборки ОС, отладки через виртуальную машину и взаимодействия с низкоуровневыми компонентами системы.

\begin{center}
б) NetBSD
\end{center}
В ходе работы были освоены процессы установки NetBSD в виртуальную машину, настройки среды сборки и перекомпиляции ядра. Основной сложностью стало корректное конфигурирование системы для вывода отладочной информации, а также поиск подходящего места в исходном коде ядра для добавления пользовательской строки. Однако сама сборка и замена ядра прошли успешно благодаря четкой документации NetBSD и логичной структуре системы. Важным итогом стало понимание особенностей работы с BSD-системами, включая управление исходным кодом и отладку на уровне ядра.


\begin{center}
\section{Список литературы}
\end{center}
1) Учебно-методическое пособие "Операционные системы" [Электронный ресурс]. URL:\\
https://press.bmstu.ru/catalog/item/8226/reader/\\
2) Статья о настройке VirtualBox [Электронный ресурс]. URL:\\
https://www.nakivo.com/blog/use-virtualbox-quick-overview/\\
3) Проект NetBSD [Электронный ресурс]. URL:\\
https://www.netbsd.org\\
4) Статья “Obtaining the sources” [Электронный ресурс]. URL:\\
https://netbsd.org/docs/guide/en/chap-fetch.html\#chap-fetch-cvs\\
5) Статья “Compiling the kernel” [Электронный ресурс]. URL:\\
https://netbsd.org/docs/guide/en/chap-kernel.html\\
6) Инструкция по сборке ReactOS [Электронный ресурс]. URL:\\
https://reactos.org/wiki/Building\_ReactOS\\
7) Инструкция по отладке ReactOS [Электронный ресурс]. URL:\\
https://reactos.org/wiki/Debugging\\


\end{document}
