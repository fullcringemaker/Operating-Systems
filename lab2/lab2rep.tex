\documentclass[a4paper, 14pt]{extarticle}

% Поля
%--------------------------------------
\usepackage{geometry}
\geometry{a4paper,tmargin=2cm,bmargin=2cm,lmargin=3cm,rmargin=1cm}
%--------------------------------------


%Russian-specific packages
%--------------------------------------
\usepackage[T2A]{fontenc}
\usepackage[utf8]{inputenc} 
\usepackage[english, main=russian]{babel}
%--------------------------------------

\usepackage{textcomp}

% Красная строка
%--------------------------------------
\usepackage{indentfirst}               
%--------------------------------------             


%Graphics
%--------------------------------------
\usepackage{graphicx}
\graphicspath{ {./images/} }
\usepackage{wrapfig}
%--------------------------------------

% Полуторный интервал
%--------------------------------------
\linespread{1.3}                    
%--------------------------------------

%Выравнивание и переносы
%--------------------------------------
% Избавляемся от переполнений
\sloppy
% Запрещаем разрыв страницы после первой строки абзаца
\clubpenalty=10000
% Запрещаем разрыв страницы после последней строки абзаца
\widowpenalty=10000
%--------------------------------------

%Списки
\usepackage{enumitem}

%Подписи
\usepackage{caption} 

%Гиперссылки
\usepackage{hyperref}

\hypersetup {
	unicode=true
}

%Рисунки
%--------------------------------------
\DeclareCaptionLabelSeparator*{emdash}{~--- }
\captionsetup[figure]{labelsep=emdash,font=onehalfspacing,position=bottom}
%--------------------------------------

\usepackage{tempora}

%Листинги
%--------------------------------------
\usepackage{listings}
\lstset{
  basicstyle=\ttfamily\footnotesize, 
  %basicstyle=\footnotesize\AnkaCoder,        % the size of the fonts that are used for the code
  breakatwhitespace=false,         % sets if automatic breaks shoulbd only happen at whitespace
  breaklines=true,                 % sets automatic line breaking
  captionpos=t,                    % sets the caption-position to bottom
  inputencoding=utf8,
  frame=single,                    % adds a frame around the code
  keepspaces=true,                 % keeps spaces in text, useful for keeping indentation of code (possibly needs columns=flexible)
  keywordstyle=\bf,       % keyword style
  numbers=left,                    % where to put the line-numbers; possible values are (none, left, right)
  numbersep=5pt,                   % how far the line-numbers are from the code
  xleftmargin=25pt,
  xrightmargin=25pt,
  showspaces=false,                % show spaces everywhere adding particular underscores; it overrides 'showstringspaces'
  showstringspaces=false,          % underline spaces within strings only
  showtabs=false,                  % show tabs within strings adding particular underscores
  stepnumber=1,                    % the step between two line-numbers. If it's 1, each line will be numbered
  tabsize=2,                       % sets default tabsize to 8 spaces
  title=\lstname                   % show the filename of files included with \lstinputlisting; also try caption instead of title
}
%--------------------------------------

%%% Математические пакеты %%%
%--------------------------------------
\usepackage{amsthm,amsfonts,amsmath,amssymb,amscd}  % Математические дополнения от AMS
\usepackage{mathtools}                              % Добавляет окружение multlined
\usepackage[perpage]{footmisc}
%--------------------------------------

%--------------------------------------
%			НАЧАЛО ДОКУМЕНТА
%--------------------------------------

\begin{document}

%--------------------------------------
%			ТИТУЛЬНЫЙ ЛИСТ
%--------------------------------------
\begin{titlepage}
\thispagestyle{empty}
\newpage


%Шапка титульного листа
%--------------------------------------
\vspace*{-60pt}
\hspace{-65pt}
\begin{minipage}{0.3\textwidth}
\hspace*{-20pt}\centering
\includegraphics[width=\textwidth]{emblem}
\end{minipage}
\begin{minipage}{0.67\textwidth}\small \textbf{
\vspace*{-0.7ex}
\hspace*{-6pt}\centerline{Министерство науки и высшего образования Российской Федерации}
\vspace*{-0.7ex}
\centerline{Федеральное государственное бюджетное образовательное учреждение }
\vspace*{-0.7ex}
\centerline{высшего образования}
\vspace*{-0.7ex}
\centerline{<<Московский государственный технический университет}
\vspace*{-0.7ex}
\centerline{имени Н.Э. Баумана}
\vspace*{-0.7ex}
\centerline{(национальный исследовательский университет)>>}
\vspace*{-0.7ex}
\centerline{(МГТУ им. Н.Э. Баумана)}}
\end{minipage}
%--------------------------------------

%Полосы
%--------------------------------------
\vspace{-25pt}
\hspace{-35pt}\rule{\textwidth}{2.3pt}

\vspace*{-20.3pt}
\hspace{-35pt}\rule{\textwidth}{0.4pt}
%--------------------------------------

\vspace{1.5ex}
\hspace{-35pt} \noindent \small ФАКУЛЬТЕТ\hspace{80pt} <<Информатика и системы управления>>

\vspace*{-16pt}
\hspace{47pt}\rule{0.83\textwidth}{0.4pt}

\vspace{0.5ex}
\hspace{-35pt} \noindent \small КАФЕДРА\hspace{50pt} <<Теоретическая информатика и компьютерные технологии>>

\vspace*{-16pt}
\hspace{30pt}\rule{0.866\textwidth}{0.4pt}

\vspace{11em}

\begin{center}
\Large {\bf Отчет о лабораторной работе №2} \\ 
\large {\bf по курсу <<Операционные системы>>} \\
\large «Загружаемый модуль ядра (драйвер)»
\end{center}\normalsize

\vspace{8em}


\begin{flushright}
  {Студент: Трофименко Д. И. \hspace*{15pt}\\ 
  \vspace{2ex}
  Группа: ИУ9 - 42Б \hspace*{15pt}\\
  \vspace{2ex}
  Преподаватель: Брагин А. В.}
\end{flushright}

\bigskip

\vfill
 

\begin{center}
\textsl{Москва, 2025}
\end{center}
\end{titlepage}
%--------------------------------------
%		КОНЕЦ ТИТУЛЬНОГО ЛИСТА
%--------------------------------------

\renewcommand{\ttdefault}{pcr}

\setlength{\tabcolsep}{3pt}
\newpage
\setcounter{page}{2} 
\begin{center}
\section*{Содержание}\label{Sect::task}
\end{center}

\begin{flushleft}
\begin{tabular}{l@{\hspace{4cm}}r}
1) Постановка задачи & \hspace{4cm} \framebox[1cm]{3} \\
2) Практическая реализация & \hspace{4cm} \framebox[1cm]{4} \\
3) Результаты & \hspace{4cm} \framebox[1cm]{5} \\
4) Выводы & \hspace{4cm} \framebox[1cm]{6} \\
5) Список литературы & \hspace{4cm} \framebox[1cm]{6} \\
\end{tabular}
\end{flushleft}

\newpage
\begin{center}
\section{Постановка задачи}
а) ReactOS
\end{center}
В  созданном  в  лабораторной  работе  № 1  рабочем дереве  ОС ReactOS создать новый модуль, который представляет собой простейший  драйвер,  совместимый  с  ОС  Windows  NT  и  ReactOS. Драйвер  должен реализовать  минимальный  набор функций,  необходимый  для  загрузки  и  выгрузки  этого  драйвера. В  функции инициализации  этого  драйвера  DriverEntry  осуществить вывод  в отладочный  лог,  используя макрос  DPRINT1, фамилии студента, выполнившего лабораторную работу.

\begin{center}
б) NetBSD
\end{center}
В виртуальной  машине  с NetBSD, созданной  в  лабораторной работе  №  1,  создать  новый  загружаемый  модуль  ядра  (loadable kernel module), реализующий простейший драйвер. Драйвер  должен содержать  минимальный  набор  функций,  необходимый  для загрузки  и  выгрузки  этого  драйвера.  В функции  инициализации  драйвера осуществить вывод в отладочный лог фамилии студента, выполнившего лабораторную работу. Список  загружаемых  драйверов  можно  получить  командой find /stand -type f, а затем найти код искомого модуля


\vspace{0.1em}


\newpage
\vspace{2em}
\begin{center}
\section{Практическая реализация}\label{Sect::res}
а) ReactOS
\end{center}
1) Используя указания из методички создал папку drivers/lab2, разместил
там файлы CMakeLists.txt для сборки и файлы lab2driver.c, lab2driver.rc [1].\\
2) Воспользовался кодами других драйверов, лежащих в папке drivers,
чтобы понять, что записывать в реализации драйвера и файле rc\\
3) Собрал новый установочный образ\\
4) Выполнил загрузку драйвера в операционной системе\\
5) Заменил ядро, перезагрузил машину, убедился, что в отладочном логе
появляется фамилия при загрузке драйвера.\\

\begin{center}
б) NetBSD
\end{center}
1) Создал файл /usr/src/sys/dev/lab2.c [1]\\
2) Добавил минималистичную реализацию драйвера из методички\\
3) Создал Makefile\\
4) Выполнил компиляцию командой make\\
5) Заменил ядро, перезагрузил машину, убедился, что в отладочном логе
появляется фамилия при загрузке драйвера.\\

\newpage

Исходный код программы представлен в листингах~\ref{lst:code1}--~\ref{lst:code2}.
\begin{figure}[!htb]
\begin{lstlisting}[language={},caption={
lab2driver.c (ReactOS).},label={lst:code1}]
#include <ntddk.h>

#ifndef NDEBUG
#define NDEBUG
#endif
#include <debug.h>
VOID NTAPI DriverUnload(IN PDRIVER_OBJECT DriverObject);
NTSTATUS NTAPI DriverEntry(IN PDRIVER_OBJECT DriverObject, IN PUNICODE_STRING RegistryPath)
{
    DPRINT1("---------------Trofimenko Dmitriy---------------\n");
    DriverObject->DriverUnload = DriverUnload;
    return STATUS_SUCCESS;
}
VOID NTAPI DriverUnload(IN PDRIVER_OBJECT DriverObject)
{
    DPRINT1("---------------Trofimenko Dmitriy---------------\n");
}



\end{lstlisting}
\end{figure}

\begin{figure}[!htb]
\begin{lstlisting}[language={},caption={
lab2.c (NetBSD).},label={lst:code2}]
#include <sys/module.h>
MODULE(MODULE_CLASS_MISC, lab2, NULL);
static int lab2_modcmd(modcmd_t cmd, void* arg) {
    printf("driver LAB2");
    return 0;
}


\end{lstlisting}
\end{figure}




\vspace{1em}
\begin{center}
\section{Результаты}

\begin{minipage}{\linewidth}
\vspace{50pt}
\centering
\includegraphics[width=1.0\linewidth]{result_reactos.jpg}
\captionof{figure}{result\_reactos}
\label{fig:result_reactos.jpg}
\end{minipage}

\begin{minipage}{\linewidth}
\vspace{50pt}
\centering
\includegraphics[width=1.0\linewidth]{result_netbsd.jpg}
\captionof{figure}{result\_netbsd}
\label{fig:result_netbsd.jpg}
\end{minipage}

\end{center}

\newpage

\begin{center}
\section{Выводы}
а) ReactOS
\end{center}
В ходе выполнения лабораторной работы были изучены основы разработки загружаемого модуля ядра (драйвера) для ReactOS, совместимого с Windows NT. Основной сложностью стала правильная реализация функции DriverEntry в файле lab2driver.c, включая корректное использование макроса DPRINT1 для вывода фамилии в отладочный журнал, а также проверка возвращаемых статусов, чтобы драйвер не вызывал ошибок при загрузке. Однако процесс интеграции драйвера в систему (через CMakeLists.txt и sc) оказался достаточно простым благодаря подробной документации ReactOS. Работа позволила получить практический опыт написания драйверов и отладки их работы через системные журналы, что важно для дальнейшего изучения разработки модулей ядра.

\begin{center}
б) NetBSD
\end{center}
В ходе выполнения лабораторной работы были изучены основы разработки загружаемого модуля ядра для NetBSD. Основной сложностью стало корректное оформление структуры модуля (использование макроса MODULE и функции modcmd), а также настройка системы сборки через Makefile, поскольку требовалось точное соблюдение путей и правил NetBSD. Однако сам процесс компиляции и загрузки модуля (make и modload) оказался достаточно простым благодаря продуманной системе управления модулями в NetBSD. Работа позволила получить практический опыт создания LKM (Loadable Kernel Module) и познакомила с особенностями вывода отладочной информации на уровне ядра в Unix-подобных системах.


\begin{center}
\section{Список литературы}
\end{center}
1) Учебно-методическое пособие "Операционные системы"  [Электронный ресурс]. URL:\\
https://press.bmstu.ru/catalog/item/8226/reader/\\




\end{document}
